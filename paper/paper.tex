\documentclass[11pt]{article}
%\documentclass[11pt,draft]{article}   % uncomment this and comment out the above line for *fast* typesetting (no images)
\usepackage{graphicx}
\usepackage{epstopdf}
\usepackage{amsmath}
\usepackage{amsfonts}
\usepackage{amssymb}
\usepackage{amsthm}
\usepackage{sistyle}\SIthousandsep{,}
\usepackage{hyperref}

\DeclareGraphicsRule{.tif}{png}{.png}{`convert #1 `dirname #1`/`basename #1 .tif`.png}

\newcommand{\mycaption}[1]{\begin{quote}{\bf Figure: } \large #1\end{quote}}



%%%% Theoremstyles
\theoremstyle{plain}
\newtheorem{theorem}{Theorem}[section]
\newtheorem{proposition}[theorem]{Proposition}
\newtheorem{corollary}[theorem]{Corollary}
\newtheorem{claim}[theorem]{Claim}
\newtheorem{lemma}[theorem]{Lemma}
\newtheorem{hypothesis}[theorem]{Hypothesis}
\newtheorem{conjecture}[theorem]{Conjecture}

\theoremstyle{definition}
\newtheorem{definition}[theorem]{Definition}
\newtheorem{question}[theorem]{Question}
\newtheorem{project}[theorem]{Project}
\newtheorem{problem}[theorem]{Problem}
\newtheorem{alg}[theorem]{Algorithm}
\newtheorem{openproblem}[theorem]{Open Problem}

%\theoremstyle{remark}
\newtheorem{goal}[theorem]{Goal}
\newtheorem{aside}[theorem]{Aside}
\newtheorem{remark}[theorem]{Remark}
\newtheorem{remarks}[theorem]{Remarks}
\newtheorem{example}[theorem]{Example}
\newtheorem{exercise}[theorem]{Exercise}
\newtheorem{suggestion}[theorem]{Suggestion}
\numberwithin{equation}{section}
\numberwithin{figure}{section}
\numberwithin{table}{section}


\textwidth = 6.5 in
\textheight = 9 in
\oddsidemargin = 0.0 in
\evensidemargin = 0.0 in
\topmargin = 0.0 in
\headheight = 0.0 in
\headsep = 0.0 in
\parskip = 0.2in
\parindent = 0.0in


\def\GL{\mathrm{GL}}
\def\PGL{\mathrm{PGL}}
\def\PSL{\mathrm{PSL}}
\def\SS{\mathcal{S}}
\def\s{S_{\rm Riemann}(X)}
\def\Z{\bf{Z}}
\def\Q{\bf{Q}}
\def\Gal{\mathrm{Gal}}
\def\Hom{\mathrm{Hom}}
\def\Ind{\mathrm{Ind}}
\def\End{\mathrm{End}}
\def\Aut{\mathrm{Aut}}
\def\loc{\mathrm{loc}}
\def\glob{\mathrm{glob}}
\def\Kbar{{\bar K}}
\def\D{{\mathcal D}}
\def\z{{\mathcal Z}}
\def\l{{\Lambda}}
\def\L{{\mathcal L}}
\def\p{{\mathcal P}}
\def\R{{\bf R}}
\def\G{{\mathcal G}}
\def\W{{\mathcal W}}
\def\H{{\mathcal H}}
\def\O{{\mathcal O}}


\title{How explicit is the Explicit Formula?}
%{Arithmetic statistics of central zeroes  of $L$-functions of the symmetric $n$-th powers of a given automorphic form}
\author{Barry Mazur and William Stein}
\begin{document}
\maketitle


%\hskip20pt ({\it Notes for our  20+20 minute talk at the  AMS Special Session on Arithmetic Statistics in San Diego, January 2013}
\vskip10pt
%\tableofcontents
\section{Introduction}
\vskip10pt

The  term `Explicit Formula' as in our title refers to the genre of formula that expresses {\it arithmetically interesting quantities} in terms of the zeroes of  a related zeta-function or $L$-function.  The first such formula appears towards the end of Bernhard Riemann's short and great paper  ``\"{U}ber die Anzahl der Primzahlen unter einer
gegebenen Gr\"{o}sse," published in the
Monatsberichte der Berliner Akademie,
November 1859.  For real numbers $X>1$ not equal to a power of a prime number  (and with some minor changes in notation and ordering  of terms) Riemann's formula is:


\begin{align*}
{\bf (*)}\ \ \ \sum_{n} {\frac{1}{n}}\pi(X^{{\frac{1}{n}}}) \ \ \ &= \\
  {\rm Li}(X) \ \ \  +& \ \ \  \left\{ \int_X^{\infty}{\frac{1}{x^2-1}}\cdot {\frac{dx}{x\log(x)}} +\log(\xi(0))\right\}\ \ \ \     - \ \ \
  \sum_\theta\left({\rm Li}(X^{{\frac{1}{2}}+i\theta})+ {\rm Li}(X^{{\frac{1}{2}}-i\theta})\right).
\end{align*}


Here,  the Riemann Hypothesis is assumed,
 the $\pm \theta$ are the imaginary parts of the nontrivial zeroes, and  $\xi(0)$ is---it seems---a scribe's error for  $\zeta(0)=1/2$.
%Riemann obtains his formula by analyzing   $\log\zeta(s)$ which is $\sum_p \log(1+p^{-s)}$  for ${\rm Re}(s) > 1$  and dealing with poles and zeroes by means of the Cauchy integral theorem
The three terms on the right correspond to, respectively:  the pole of $\zeta(s)$ at $s=1$, the trivial zeroes of $\zeta(s)$, and the nontrivial zeroes of $\zeta(s)$.

[[William: we need to address the fact that the above
formula is somehow nonsense, right?     See worksheets/2016-06-08-150036-very-explicit-zeta-formula.sagews]]

 Subsequently, there have been many versions of Explicit Formulas, one such notable variant  proved by Hans von  Mangoldt, in his 1895 article  ``Zu Riemann's Abhandlung `\"{U}ber die Anzahl der Primzahlen unter einer gegebenen Gr\"{o}sse'", Journal f\"{u}r die reine und angewandte Mathematik.

In celebration of Riemann's formula  we will discuss some computations and open questions related to the application of the Explicit Formula to the arithmetic of elliptic curves. This follows constructions in a letter of Peter Sarnak to one of the authors (\cite{S}).

A cartoon version of the genre of `Explicit Formula' might be described as follows:

 {{\bf (**)} \ \  ${\rm{\it Sum\ of\ local\ data\  }}  \ = \  {\rm{\it Global\ datum}}\ +\  {\rm{\it  Easy\  error\ term}}\ +\  {\rm{\it  Oscillatory\ term}},$ }


 \noindent where each  term of the formula  is taken as a function of a `cutoff' real value $X$, and the ordering of the terms is in accord with {\bf(*)} above.
  The {\it Sum of local data} that appears in most of the `Explicit Formulas' of analytic number theory is often given, i.e., by a partial sum $\nu(X)\cdot\sum_{p<X}G(p)$ of locally defined {\it arithmetically interesting} quantities $G(p)$ attached to prime numbers $p$, summed to up to some cutoff
value $p<X$  (and normalized for convenience by an elementary (continuous) multiplicative factor $\nu(X)$).

   Usually the {\it Global datum} is computed by knowing the order of specific zeroes (or poles) at specified values of the complex variable $s$  of relevant (global) $L$-functions.  For example, the formula {(*)} gives the contribution of the pole at $s=1$ of $\zeta(s)$.  Often (perhaps only conjecturally)  the  {\it Global datum} is the `dominant term'    on the right-hand-side of the equation.  Sometimes, as will be the case  of the examples to be described below, the roles are reversed and one takes the Global datum as the object being studied.  That is, we view it as ${\rm{\it Global\ datum}}\ = \  {\rm{\it Sum\ of\ local\ data\  }}\ -\  {\rm{\it  Easy\ error\ term}}\ -\  {\rm{\it  Oscillatory\ term}}.$

   In some instances,  the sought-for Global datum is  (conjecturally) constant, independent of the cutoff $X$---in fact, the mean of {\it Sum of local data}---and also an integer, in which case a close approximation to each of the other three terms (for some specific value of $X$) would give a decisive answer for the value of the Global datum.


 \subsection{\bf An example: }  Consider a non-CM elliptic curve $E$ over ${\Q}$ of conductor $N$  uniformized by a newform $f_E$ (of conductor $N$ and of weight two) with Fourier expansion  $f_E(q) = \sum_{n\ge 1}a_n(E)q^n$. Form the following {\it Sum of local data}:


    $$ {\mathcal D}_E(X): = {\frac{\log\ X}{\sqrt X}}\sum_{p \le X}{\frac{ a_{E}(p)}{\sqrt p}},$$

    \noindent where (assuming the Riemann Hypothesis for the $L$-function attached to  $f_E$) the explicit formula gives us:

    $$ {\mathcal D}_E(X)  \ = \ 1-2r_E  \ +\  {\rm{\it  Easy\  error\ term}}\ +\  \sum_{|\gamma| \le T}{\frac{X^{i\gamma}}{{\frac{1}{2}}+i\gamma}}$$

[[TODO: explain $T$, e.g., take limit as $T\to\infty$
or make error term depend on $T$.]]

     Here, the `global datum'  is $1-2r_E$ where  $r_E$ is the analytic rank of $E$  (i.e., the order of vanishing of  $L(E,s)$ at the central point; conjecturally this will be the Mordell-Weil rank of $E$). The summation is taken for $\gamma$ ranging over the imaginary parts of the nontrivial zeroes of the $L$-function attached to $E$.

   The   values of  ${\mathcal D}_E(X)$ achieve a  limiting distribution $\mu_E$ (with respect to multiplicative Haar measure $dx/x$) with {\it mean} $M$ equal to $1- 2r_E$  and variance  $V$  equal to $$\lim_{T \to \infty} \sum_{|\gamma | \le T}{\frac{1}{{\frac{1}{4}} + \gamma^2}}.$$

   Under further conjectures (\cite{S})  the measure $\mu_E$  is symmetric
about its mean $1-2r_E$, it is smooth, and has support all of $(-\infty, +\infty)$. It is natural to view this in the context of Chebyshev biases; i.e., to interpret---when $r_E$ is positive---the minus sign in $-r_E$ as a measure of the bias that the values $a_p(E)$ have to being negative. This type of {\it bias} in the arithmetic statistics of elliptic curves  hearkens back to the early work of Birch and Swinnerton-Dyer \cite{S}, and in the context we are considering was  first written down by Peter Sarnak  in the spirit of the classical Chebyschev bias, and 'prime races.'   An `Explicit Formula' account of this in the context of  prime numbers in different arithmetic progressions can be found in {\cite{GM}}.  See also \cite{R-S}, \cite{S}
which fits into this Explicit Formula format, as does  work of  Martin-Watkins,\ Fiorilli,\ Bober-Mestre-Odlyzko,\ Conrey-Snaith, \ Spicer, and others.




   For this volume in honor of Bernhard Riemann,  we will offer computations related to this, and one  other interesting case  that does depend on these `finer conjectures',  advertise the conjectures themselves, and  the need for  computational projects regarding  such problems.  We restrict ourselves to applications to elliptic curves, but we take this only as a basic (important) example of a fuller story for general motives.  We have  future plans for a web-accessible resource\footnote{See \url{http://wstein.org/papers/2016-explicit/}}: a repository of some of the numerics for the cases related to elliptic curves that interest us.


 One problem we consider is related to the question---given an elliptic curve over the rational numbers and letting  $p$ range through prime numbers---of how often  $p+1$ is an {\it over-count} or an {\it under-count} for the number of rational points on the curve modulo $p$? The rough answer is 50/50, but there can be a `bias'.


 We offer no new theoretical results but, as mentioned, we use this occasion to exhibit computations and recall some interesting recent work and conjectures  (of other people)  that might warrant more such computations and that raise a host of questions, both theoretical and computational.   For example, to do some systematic numerical computations related to an elliptic curve $E$ attached to a newform $f_E$  (along the lines of what has already been done in this paper)  it would be very useful to have a much larger data set  of the arithmetic function  $$n \mapsto r_E(n)$$
 where $r_E(n)$ is the order of vanishing at $s=1$ of the $L$-function of the automorphic forms ${\rm symm}^n f_E$ for odd values of $n$.  Regarding this arithmetic function, aside from having control of the parity of   $r_E(n)$  (e.g., see \cite{DMW})  hardly anything else is known. Nor do we (at least, the authors of this paper)  yet have enough
 experience---when $E$ has no complex multiplication---even to formulate a proper conjecture.

 We might also mention that when making these numerical experiments one seems to be in a situation  that is not entirely dissimilar from the type of slightly annoying mismatch between conjecture and data that one encounters in more traditional studies of Mordell-Weil statistics  that was the subject of the survey article \cite{bmsw:bulletins}, and the more recent \cite{ecdb:height}.  But this may be unavoidable, given that even  the so-called  `easy error term' in the explicit formula may tend to zero rather slowly.

 We should say at the outset that for simplicity, and sometimes for necessity, we'll be assuming GRH throughout---without any further mention. In fact, at times we'll also be assuming ({\it with} explicit warning) some further conjectures. Given our current state of knowledge, it would be interesting enough to work  conditionally on some reasonable conjectures that include or extend the Grand Riemann Hypothesis.  In fact, the search for an answer to such questions  might give  motivation for the refinement of conjectures---interesting in their own right---that complement the Riemann Hypothesis.


 \section{\bf Sarnak distributions for the elliptic curve $E$ relative to the weighting function $V$}

  In a letter  \cite{S}  to one of us (to B.M.) Peter Sarnak considered a broad array of statistics related to local arithmetic data associated to modular forms,  and, in particular, to elliptic curves over ${\Q}$.


For $p$ a prime, write

\begin{equation}
{\frac{a_E(p)}{\sqrt p}}: = \   \alpha_p+\beta_p,
\end{equation}

with $\alpha_p= e^{i\theta_p}$ and  $\beta_p= e^{-i\theta_p}$
and
\begin{equation}
  \theta_p \in [0, \pi].
\end{equation}


Our basic data consists of the function

\begin{equation}\label{data}
p \ \mapsto\ \theta_p
\end{equation}

To have some vocabulary to deal with its statistics, consider
$$U_n(\theta) : = {\frac {\sin((n+1)\theta)}{\sin(\theta)}},$$
so we have:
  $$\frac{a_E(p)}{\sqrt p} = U_1(\theta_{E,p}).$$  Note that the set $\{U_n\}$ for $n=0,1,2,\dots$ forms an orthonormal basis of the Hilbert space $L^2[0,\pi]$  with the
 inner product

 $$\langle f, g\rangle :={\frac{2}{\pi}} \int_0^\pi f(\theta)g(\theta)\sin^2(\theta)d\theta.$$

For $V(\theta)$ a smooth function on $[0,\pi]$, write $V=\sum_{n=0}^{\infty} c_nU_n$ with $c_n: = \langle V, U_n\rangle$.

Let us define the ``$V$-weighted average of the data"
\begin{equation}
D_V(X):= {\frac{\log X}{\sqrt X}}\sum_{p \le X} \ V(\theta_p)
\end{equation}


 so that   $${\mathcal D}_E(X)=  {\mathcal D}_{U_1}(X).$$


Just to cut down to the essence as rapidly as possible, and just for the present paper:

\begin{definition} Say that our data (\ref{data}) has {\bf `Explicit Formula' statistics} if there is a sequence of non-negative integers $\{r_n\}_n$  for $n=1,2,3, \dots$ such that for all smooth functions $V(\theta)$ as above with $c_0=0$,
\begin{itemize}
\item
possesses a limiting distribution{\footnote{\label{footnote:statdist} Recall that, as in subsection    \ref{statdist} above,  $S_V(x)$ {\bf possesses a limiting distribution $\mu_V$ with respect to the multiplicative measure $dx/x$} if for continuous bounded functions $f$ on ${\bf R}$ we have:
\begin{equation}
\lim_{X \to {\infty}}\ {\frac{1}{\log X}}\int_0^Xf(S_V(x))dx/x \ = \ \int_{\bf R}f(x)d\mu_V(x).
\end{equation}}}
 $\mu_V$ with respect to the multiplicative measure $dX/X$,
\item  $\mu_V$ has support on all of ${\bf R}$, is continuous and symmetric about its mean, ${\mathcal E}(S_V)$, and
\begin{equation}\label{eqnmean}
{\mathcal E}(S_V)\ = \ -\sum_{n=1}^{\infty}  c_n\big(2r_n+(-1)^n\big).
\end{equation}
\end{itemize}
\end{definition}
We will refer to $\mu_V$  as the {\bf Sarnak Distribution  (for the elliptic curve $E$ relative to the weighting function $V$)}.
One can also compute---given some plausible conjectures---the behavior of the {\bf variance}  (i.e., the measure of fluctuation of the values of $S_V(X)$ about the mean) as well; the variance is defined by the formula  $${\mathcal V}(S_V): = {\mathcal E}\big([S_V  - {\mathcal E}(S_V)]^2\big).$$


\begin{remark}  If some standard conjectures{\footnote{that (for $n=1,2,\dots$) the $L$-functions of the symmetric $n$-th powers of the elliptic curve, \begin{equation}
L(s, E, {\rm sym}^n): = \prod_p\prod_{j=0}^n(1- \alpha_p^{n-j}\beta_p,^jp^{-s})^{-1},
\end{equation} have analytic continuation   to the entire complex plane satisfying a standard function equation (and one can relax analyticity and require merely an appropriate meromorphicity hypothesis) and that they be holomorphic and nonvanishing up to $Re(s) =1/2$ (i.e., GRH).  The integer $r_n$ (for $n=1,2,\dots$)  is then the multiplicity of the zero of $L(s, E, {\rm sym}^n)$ as $s=1/2$. \vskip20pt }} and some non-standard conjectures{\footnote{LI(E); see  \ref{S}, \ref{F} [[TODO:expand on this mysterious note -- addressed later "Fiorilli calls his..."]]}}  hold, then our data (\ref{data}) would indeed have {\it `Explicit Formula' statistics}; for details, see \cite{S}.  The integers $r_n$, which by footnote  \label{footnote:statdist} are (conjecturally) the orders of vanishing of specific $L$-functions at their central points, are expected to have the large preponderance of their values equal to  $0$ or $1$, depending on the sign of the functional equation satisfied by the $L$-function to which they are associated,  so the {\it mean} for  a given $V$ as computed by equation (\ref{eqnmean}) stands a good chance of being finite.
\end{remark}




\section{The Letter of Peter Sarnak}  In a letter  \cite{S}  to one of us (to B.M.) Peter Sarnak sketched reasons for the statements made about the two formats for sums of local data that we  introduced above, and indeed, for large class of such formats. As we  understand it, the computations in that letter was, at least in part, the fruit of conversations with Andrew Granville and also an outgrowth of \cite{R-S}. We are grateful for that letter, and for  illuminating discussions with    Granville, Rubinstein, and Sarnak.  Assuming a list of standard conjectures about the behavior of $L$-functions, together with some very plausible but less standard conjectures, Sarnak begins by showing---as we mentioned above---that (conditional on standard conjectures)  the following  'Sum of local data'   $$ {\mathcal D}_E(X): = {\frac{\log\ X}{\sqrt X}}\sum_{p \le X}{\frac{a_{\mathcal E}(p)}{\sqrt p}},$$ has a limiting distribution with {\it mean} equal to $1- 2r_E$; the {\it variance} of this limiting distribution  is the sum of the squares of the reciprocals of the absolute values of the nonreal zeroes of the $L$-function of $E$. The argument for these (and related) facts follows Mike Rubenstein's and Peter Sarnak's line of reasoning in the article {\it Chebyshev's Bias} [\ref{R-S}]. For another expository account of number theoretic issues related to biases, see [\ref{GM}]. Similar reasoning works for other formats, including the {\it raw} sum of local data as will be depicted in our graphs below; i.e.,  $$\Delta_E(X):= {\frac{\log\ X}{\sqrt X}}\big(\#\{ {p \le X};\ a_E(p) > 0\} \ - \ \#\{ {p \le X};\ a_E(p) < 0\}\big),$$ which  (given reasonable conjectures, and guesses)  one discovers to have infinite {\it variance} so whatever bias we will be seeing in our finite stretch of data will eventually wash out{\footnote{ All this is specific to elliptic curves $E$ with no complex multiplication, as our examples below all are. The non-finiteness of the variance is related to the fact that the (expected) number of  zeroes---in  intervals  $(1/2, i/2+iT)$ ($T > 0$)---of the $L$ function of the $n$-th symmetric power of the newform $f_E$ attached to  $E$   grows at least linearly with $n$.}}.





     [[never use (*) for formulas; let latex do its thing)]]


 {\bf The bias of undercounts versus overcounts  (\cite{S})}:

 Regarding `biases,'  thanks to the recent resolution \cite{} [[todo: in https://webusers.imj-prg.fr/~michael.harris/SatoTate/notes/Introduction.pdf]] of the Sato-Tate Conjecture in this context, one knows that---roughly---half the Fourier coefficients  $a_E(p)$ are positive and half negative (for non-CM curves). That is,  the ratio

 $${\frac{\#\{p < X \ | \ N_E(p) < p+1 \}} {\#\{p < X \ | \ N_E(p) > p+1 \}}}\ =\ {\frac{\#\{p < X \ | \ a_E(p) > 0 \}} {\#\{p < X \ | \ a_E(p) < 0 \}}}$$

 tends to $1$ as $X$ goes to infinity. Moreover, the numbers of positive values and negative values look very close to each other:
\vskip20pt


\begin{center}
\begin{tabular} {r | c | c | c | r}\hline
Curve & Rank & Negative $a_E(p)$ for $p<10^9$ & Positive $a_E(p)$ for $p<10^9$ & Difference\\ \hline\hline
11a    & 0      & 25422268       & 25423101     &   -833 \\ \hline
14a    & 0      & 25422229       & 25421074     &   1155 \\ \hline
128b   & 0      & 25420641       & 25425608     &   -4967 \\ \hline
816b   & 0      & 25424848       & 25421229     &   3619 \\ \hline
2379b  & 0      & 25417900       & 25427007     &   -9107 \\ \hline
5423a  & 0      & 25420479       & 25425242     &   -4763 \\ \hline
29862s & 0      & 25420525       & 25425197     &   -4672 \\ \hline
37a    & 1      & 25423396       & 25422448     &   948 \\ \hline
43a    & 1      & 25421536       & 25424196     &   -2660 \\ \hline
160a   & 1      & 25424446       & 25421488     &   2958 \\ \hline
192a   & 1      & 25418843       & 25426859     &   -8016 \\ \hline
2340i  & 1      & 25425512       & 25419660     &   5852 \\ \hline
10336d & 1      & 25421245       & 25423628     &   -2383 \\ \hline
389a   & 2      & 25427014       & 25418738     &   8276 \\ \hline
433a   & 2      & 25425902       & 25419896     &   6006 \\ \hline
2432d  & 2      & 25423818       & 25421900     &   1918 \\ \hline
3776h  & 2      & 25422350       & 25422750     &   -400 \\ \hline
5077a  & 3      & 25426985       & 25418831     &   8154 \\ \hline
11197a & 3      & 25429098       & 25416702     &   12396 \\ \hline
\end{tabular}
\end{center}



To deal with the more delicate structure of these statistics, consider the following `Sum of local data'
  $${\frac{\log X}{{\sqrt X}}}\sum_{p\le X}\gamma_E(p)$$  where $\gamma_E(p)=0$ if $p$ is a bad or supersingular prime for $E$ and is otherwise is $+1$ if $E$ has less that $p+1$ rational points over ${\bf F}_p$; and $\gamma_E(p) = -1$ if $E$ has more that $p+1$ points.  Then this sum, which will be denoted $\Delta_E(X)$, measures exactly the difference between over-count and under-count{\footnote{  Ralph Greenberg has raised the following question: if $E$ and $F$ are elliptic curves such that $\gamma_E(p)=\gamma_F(p)$ for all (or almost all?) $p$, are $E$ and $F$ necessarily isogenous?  TODO: I asked Ralph and he said that there is now a paper by some Koreans about this, with a nice conditional answer, evidently
  following approach that I suggested.}}. % For some contrast, consider these two variant sums of local data, and their corresponding Explicit formulas, as  proved in \cite{S}

 The  mean of $\Delta_E(X)$ is  (conjecturally) \begin{equation*}
{\frac{2}{\pi}}- {\frac{16}{3\pi}}r_E \ \ \ + \ \ \  {\frac{4}{\pi}} \sum_{k=1}^{\infty}  (-1)^{k+1}\Bigl({\frac{1}{2k+1}} + {\frac{1}{2k+3}}\Bigr) r_E({2k+1}).
\end{equation*} where $$r_E(n):= \ r_{f_E}(n)\ = \ {\rm the\ order\ of\ vanishing\ of\ }L(symm^nf_E, s)\ {\rm at}\ s=1/2,$$ with $f_E:=$ the newform of weight two corresponding to the elliptic curve $E$; and where we have normalized things so that $s=1/2$ is the central point. {\bf NOTE:} For a discussion of the numerics of the values $r_E({2k+1})$, see Section {\ref{highord}} below.

  %  \subsection{The Oscillatory term}\label{osc}
  % The analogous oscillatory term for the classical Riemann zeta function  has an extensive literature. See, for example \cite{G}, \cite{Fu} and the bibliography there. Here are some pictures to convey a sense of how our $S_E(X;T)$ behaves, at least in the currently computable range which (roughly) allows $T$ to be only as high as  $10^4$.
\subsection{The bias between under-counts and over-counts}
  We will assume that our data has `Explicit Formula' statistics, and---copying Sarnak ({\cite{S}})--- apply this to the question we began with, i.e., what is the ``bias" in the race between under-counts and over-counts?

$$\Delta_E(X):={\frac{\log X}{\sqrt X}}\big(\#\{ p < X\ | \ N_E(p) < p+1\}\ - \ \#\{ p < X\ | \ N_E(p) > p+1\}\big).$$


Let $H(\theta)$ be the Heaviside function, i.e., the function with value

\begin{equation}
H(\theta) \ = \ +1
\end{equation}
 for $\theta \in [0, \pi/2)$ and  $-1$ for $\theta \in [\pi/2, \pi)$.  So
\begin{equation}
\Delta_E(X) = {\frac{\log X}{\sqrt X}}\sum_{p\le X} H(\theta_p)
\end{equation}


For $n \ge 0$, set


\begin{equation}
c_n(H)  \ = \ \langle H, U_n\rangle \ = \ {\frac{2}{\pi}}\big[\int_0^{\pi/2}U_n\sin^2\theta d \theta - \int_{\pi/2}^{\pi}U_n\sin^2\theta d \theta \big]
\end{equation}


which is $0$ if $n$ is even and $$(-1)^{(n-1)/2}{\frac{2}{\pi}}\big[{\frac{1}{n}} + {\frac{1}{n+2}}\big]$$ if $n$ is odd.



For $N \ge 1$ let

\begin{equation}
H_N(\theta): = \ \sum_{n=1}^Nc_n(H)U_n(\theta)
\end{equation}


So $H_N$ is a smoothed out version of $H(\theta)$ and $H_N(\theta) \to H(\theta)$ as $N $ tends to infinity.  Thus

\begin{equation}
S_N(X): = S_{H_N}(X) = \ {\frac{\log X}{{\sqrt{X}}}}\sum_{p \le X}H_N(\theta_p)
\end{equation}


is a smoothed out version of

\begin{equation}\label{smooth}
S(X): = S_{H}(X) = \ {\frac{\log X}{{\sqrt{X}}}}\sum_{p \le X}H(\theta_p)
\end{equation}

Therefore, by formula (\ref{eqnmean}), we would have:

\begin{equation}\label{early}
{\mathcal E}(S_N)\ = \ {\frac{8}{3\pi}}(1-2r) + {\frac{2}{\pi}} \sum_{k=1}^{N}  (-1)^{k+1}\big[{\frac{1}{2k+1}} + {\frac{1}{2k+3}}\big]\big(2r_E(2k+1)-1\big).
\end{equation}


Now one does have  parity information concerning the arithmetic function $n \mapsto r_E(n)$. For a detailed study of the root numbers of $L$-functions of symmetric powers of an elliptic curve, consult \cite{DMW}.
 For $n \ge 1$ let $ \nu_E(n) \in \{0,1\}$ be (zero or one) such that  $ \nu_E(n) \equiv r_E(n)$ modulo $2$. Let $s_E(n)$ be the non-negative integer such that:
 $$r_E(n) = \nu_E(n) + 2s_E(n)$$  (for $n\ge 3$, odd).
Thus if the multiplicity of order of vanishing at the central point $s=1/2$ of the odd symmetric $n$-th power $L$-functions attached to $E$ (for $n \ge 3)$ were never greater than  $1$, and hence entirely dictated by parity, then the conjectured mean, ${\mathcal E}(S_N)$, would be equal to
\begin{equation}\label{min}
{\mathcal T}_E^{\{N\}}\ := \ {\frac{8}{3\pi}}(1-2r_E) + {\frac{2}{\pi}} \sum_{k=1}^{N}  (-1)^{k+1}\big[{\frac{1}{2k+1}} + {\frac{1}{2k+3}}\big]\big(2\nu_E(2k+1)-1\big).
\end{equation}

  Now consider the limit:
   $${\mathcal T}_E: = \lim_{N\to \infty}{\mathcal T}_E^{\{N\}}. $$
\vskip20pt
%\begin{project} Check if all the possibilities for parity as given in \cite{DMW} leads, in fact, to convergent values of ${\mathcal T}_E$.  Work out those values. E.g., In \cite{DMW} one reads that for $n$ odd and $E$ semistable, the parities of $symm^nE$ are all the same;  i.e., independent of (odd) $n$.

{\bf Note:}  in the semistable case, $${\mathcal T}_E = {\frac{8\pm 2}{3\pi}} -{\frac{16}{3\pi}}r_E,$$ where the sign depends on whether  $\nu_E(2k+1)$ is $1$ or $0$.%\end{project}
\vskip20pt


Put $${\z}_E^{\{N\}}:= {\frac{2}{\pi}}\sum_{k=1}^{N}  (-1)^{k+1}\big[{\frac{1}{2k+1}} + {\frac{1}{2k+3}}\big]\big(4s_E(2k+1)\big).$$

\vskip20pt
\noindent {\bf Questions:} Does the limit, $${\z}_E: = \lim_{N\to \infty}{\z}_E^{\{N\}} $$  exist? Does it converge to a finite value?  If so, then the conjectured mean would be:
$${\mathcal E}_E =  {\mathcal T}_E \ + \ \z_E.$$   Is $s_{2k+1}$ bounded?  Is  the set of positive integers $k$ such that  $s_{2k+1} \ne 0$ of {\it density zero} set of positive integers $k$?    Is that set finite?



 Some data for higher order of vanishing for symmetric powers is given in the article of Martin and Watkins \cite{M-W}. The following table is taken from their article:


\hskip160pt\begin{tabular} {l | r r}\hline
$E$ & $k$ & $s_{2k+1}$\\
\hline\hline
$2379b$ & 1 & 2 \\
\hline
$5423a$ &  1 & 2   \\
\hline
$10336d$ &  1 & 2  \\
\hline
$29862s$ &  1 & 2  \\
\hline
$816b$ &  2 & 1  \\
\hline
$2340i$ &  2 & 1  \\
\hline
$2432d$ &  2 & 1  \\
\hline
$3776h$ &  2 & 1  \\
\hline
$128b$ &  3 & 1  \\
\hline
$160a$ &  3 & 1  \\
\hline
$192a$ & 3 & 1  \\
\hline
\end{tabular}

\vskip40pt


\vskip10pt
\section{Recent work and further  questions}
\begin{itemize} \item {\it The relationship between bias and unbounded rank: the work of Fiorilli}\label{Fi}


In the work of Sarnak and Fiorilli, another measure for understanding `bias behavior' is given by what one might call {\bf the percentage of positive  support} (relative to the multiplicative measure $dX/X$). Namely [[define $\delta(x)$!]]:
$${\mathcal P} ={\mathcal P}_E:=  \lim {\rm inf}_{X\to \infty}{\frac{1}{\log X}}\int_{2\le x \le X; \delta(x)\le 0}dx/x$$
$$=   \lim {\rm sup}_{X\to \infty}{\frac{1}{\log X}}\int_{2\le x \le X; \delta(x)\le 0}dx/x$$
 \vskip20pt

  It is indeed a conjecture, in specific instances interesting to us, that these limits ${\mathcal E} $ and ${\mathcal P}$  exist.
   \vskip20pt

   The standard conjecture (that we have been making all along) is GRH. But here, one includes the further conjecture (given in Sarnak's letter, and the article of Fiorilli) that the the set of nontrivial complex zeroes of the relevant $L$-function $L(E,s)$ with positive imaginary part  is a set of complex numbers that are {\it linearly independent} over ${\Q}$. Such a conjecture Rubenstein and Sarnak refer to in \cite{R-S} as the {\it Grand Simplicity Hypothesis} (GSH).  Fiorilli calls his version of it  {\it Hypothesis LI(E)}.  For recent, somewhat related, work on such linear independence questions, see \cite{M-N}.   Fiorilli, following the work of Sarnak,  proves:

   \begin{theorem} Assume GRH and LI(E). Then the following two statements are equivalent:
   \begin{enumerate} \item  The set of (analytic) ranks $\{r_E\}_E$ ranging over all elliptic curves over ${\Q}$ is {it unbounded}.
   \item  The  l.u.b of the set of  {\it percentages of positive support}  $\{{\mathcal P}_E\}_E$ is equal to $1$.\end{enumerate}\end{theorem}
\item{\it The relationship between bias and bounding the rank: the
work of Bober}  In \cite{B}, Jonathan Bober  builds on work
of Odlyzko and Mestre to establish a conditional
upper bound on the ranks of various known elliptic curves of
(relatively) high Mordell-Weil rank, notably Noam Elkies'  elliptic
curve $E_{28}$ for which $28$ linearly independent rational points
have been found; Bober shows, conditional on the Birch-Swinnerton-Dyer
conjecture and GRH, that the Mordell-Weil rank of $E_{28}$ is either
$28$ or $30$ (subsequently, Jamie Weigandt used the same approach to
verify that the rank is indeed 28). He does this by a nice `bias'
computation using the
Explicit Formula.

\item Simon Spicer has further built on
Bober's work to create an algorithm for quickly bounding the analytic
ranks, which is used on hundreds of millions of elliptic
curves in \cite[\S 2.3]{ecdb:height}.

\centerline{to be inserted}
\vskip10pt
\item {\it Further questions}

 In summary, given the conjectures discussed, the {\it theory of the means} of the  weighted sums of local data we have been examining related to a non-CM elliptic curve $E$ is determined by the orders of vanishing at the central point of the $L$-functions of the symmetric powers of the modular eigenform attached to $E$: and conversely: knowledge of the means of all such weighted sums determines (conjecturally, of course) all those orders of vanishing; i.e., the arithmetic function  $k \mapsto r_E(2k+1)$ (cf. (2.6) above).
\begin{itemize}\item
Is $r_E(2k+1)\ge 2$ for  only a set of values of $k$ of density $0$?
 \item  Is  $r_E(2k+1)\ge 2$ for all but finitely many $k$'s?  \item Which weighted sums of local data have finite mean?
\item  Is there an  {\it effective version}  of  the conjecture LI(E)?\  I.e., can we find  an explicit  positive function $\Phi(H,T)$ such that for every linear combination of the form $$\sum_{j=1}^{\nu} \lambda_j\gamma_j$$ with the $\lambda_j$ 's rational numbers of height $< H$ and the  $\gamma_j$ 's  positive imaginary parts $<T$ of the complex zeroes of the $L$ function $L(E,s)$, we have  an inequality of the form $$\left|\sum_{j=1}^{\nu} \lambda_j\gamma_j\right|  > \Phi(H, T)?$$

[[idea -- instead just define $\Phi(H,T)$ to be the min
of the finitely many possible linear combinations, and ask
about the behavior of $\Phi$.  Is there some clever algorithm
to get information about this?]]
\end{itemize}\end{itemize}



\end{document}


  \centerline{ $E = 11a$}
 \vskip10pt
 \includegraphics[width=0.9\textwidth]{plots/OSC11.pdf}

  \centerline{ $E = 37a$}
 \includegraphics[width=0.9\textwidth]{plots/OSC37.pdf}
%\vskip20pt
 %\includegraphics[width=1.0\textwidth]{plots/OSC.pdf}
   \vskip20pt
  \centerline{ $E = 389a$}
\vskip20pt
 \includegraphics[width=0.9\textwidth]{plots/OSC389.pdf}



 From these examples, one might imagine that for 'most arguments $X$' the range of actually achieved values of $S_E(X)$  may be even more restricted than Sarnak's suggestion, i.e., $o(\log X)$ as quoted above.  On the one hand it would be interesting to frame (plausible)  conjectures that predict  the rate of growth of  ${\frac{1}{\log X}}S_E(X)$  more precisely. For example, see the graphs in Subsection \ref{osc} below.  Also, even if  the function $X \mapsto S_E(X)$ is in fact unbounded, it might be the case that it spends most of its time having a very restricted upper bound for its values.  To study this, let us consider the distribution of values of $S_E(X,T)$ for any fixed $(X,T)$ with $T$ large.



%%%%
\subsection{Distributions of values }

Let ${\R}_{>0}$ be the multiplicative group of positive real numbers, and ${\R}$ the additive group of reals. For $I \subset {\R}_{>0}$ a Haar measurable set, let $|I|$ denote a Haar measure.  Let ${\SS}:{\R}_{>0} \to {\R}$ be a real-valued Lebesgue-integrable function.  Fixing $I \subset {\R}_{>0}$ a subset  of finite measure,  for every measurable subset $J\subset {\R}$, form the probability measure on ${\R}$
$$J\mapsto \mu_{{\SS},I}(J): = {\frac{|I\cap {\SS}^{-1}(J)|}{|I|}}.$$  So, $\mu_{{\SS},I}(J)$ is the {\it probability that the function ${\SS}$ achieves a value in the range $J$ over the gamut of arguments in $I$.}   Say that ${\SS}$ has a {\bf distribution of values} if, for $X > 0$ setting $I_X= (0,X]$, the limit  $$\mu_{{\SS}}:= \lim_{X \to \infty}\mu_{{\SS},I_X}$$ exists. These  definitions are particularly relevant to the oscillatory terms ${\SS}(X):= S_E(X)$ that we are currently studying. The data seems to indicate convergence to a limiting distribution (the  mean  value being $0$) with a strikingly small (variance, or equivalently: strikingly small) standard deviation of values.
\vskip20pt
Here, then, are some pictures  of what seems to be data `converging' to a limiting  distribution $\mu_E$ of the values of the oscillatory terms $S_E(X)$ for a few elliptic curves $E$:
  \vskip10pt
  \centerline{ $E = 11a$}
 \vskip10pt
 \hskip100pt\includegraphics[width=0.6\textwidth]{plots/bite11.pdf}
    \vskip10pt
  \centerline{ $E = 37a$}
  \vskip10pt
 \hskip100pt\includegraphics[width=0.6\textwidth]{plots/bite37.pdf}
    \vskip10pt
       The red curve is the normal distribution with mean $0$ and standard deviation given by that of the data.
   \vskip10pt
    {\bf  Note: } Conditional on the conjecture $LI(E)$ (see section \ref{Fi} below) $\mu_E$ exists (see \ref{S}).

     It is interesting  to compare $\mu_E$ to the limiting distributions connected to the bias of nonresidues to residues mod $q$, as in \cite{R-S}. There one has the added feature that these limiting distributions themselves tend to the normal distribution as the modulus q tends to infinity.
   %\includegraphics[width=0.9\textwidth]{plots/bite389.pdf}
 %\vskip30pt
  \centerline{ $E = 389a$}
  \vskip10pt
\hskip100pt \includegraphics[width=0.6\textwidth]{plots/bite389.pdf}

 %If so, we define the {\bf  mean  value of ${\SS}$} and the {\bf standard deviation (of values) of ${\SS}$} to be the mean and standard deviation (respectively) of the limiting normal distribution $\mu_{{\SS}}$.

   {\bf Definition:}  The {\bf bite}, $\beta_E$, of the oscillatory term $S_E(X)$ is the standard deviation of the   distribution $\mu_E$ of values of $S_E(X)$.

    {\bf  Note: } Conditional, again,  on standard conjectures  one can show that the bite of $\mu_E$  grows like $c\cdot \log {\rm cond}(E)$ for some constant $c$; see \ref{S}. it is tempting to think of rescaling $\mu_E$ to have a convergent `rescaled bite' as ${\rm cond}(E)$ tends to infinity, and to ask whether (after such a rescaling) these distributions converge to the normal distribution.

 Here are a few examples comparing the bite to the conductor. We also compare this data this to the quotient $$\lambda_E:={\frac{ \log {\rm cond}(E)}{\beta_E}},$$ and to the Mordell-Weil rank $r_E$:



  \begin{tabular}{llllllllll}
  $E$   &\ & 11a & 37a & 389a & 431b1 & 443c1 & 5002c1 & 5021a1 & 5077a\\
\  &\ &\  &\  &\  &\ &\ &\ &\ &\ \\
  $\beta_E \approx $  &\ & 0.5 & 0.61 & 0.89 & 1.38 & 1.40 & 1.57 & 1.94 & 1.19\\
  $\lambda_E  \approx$  &\ & 4.8 & 5.9 & 6.7 & 4.3 & 4.4 & 5.4 & 4.4 & 7.1\\
  $r_E =$   &\ & 0 & 1 & 2 & 0 & 0 & 0 & 0 & 3 \end{tabular}
\vskip10pt
%\begin{project} Continue the computations above to be able to get good approximations to the absolute constant $c$. \end{project}

  But there is a finer structure to the behavior of the oscillatory term. For that, one must zoom in and focus attention to the values of $X$ that are close to powers of prime numbers. We will now do that.


 \subsection{ The Gibbs Phenomenon in the oscillatory term} The Explicit Formula for $D_E(X)$ tells us that we might well expect discontinuities of the function $S_E(X)$ for prime power values of $X$. The analogous question has been  examined in the case of the classical Riemann zeta-function.  Here is a brief resum{\'e} of information one finds about this in the literature. Let $$S_{\rm Riemann}(X): = \sum_{|\gamma|<X}X^{i\gamma}/i\gamma,$$ where $\gamma$ ranges through the nontrivial zeroes of $\zeta(s)$, the Riemann zeta-function. This oscillatory term has been  embedded in what one might call a `Lerch spectral zeta function,' defined by the Dirichlet series:

 $$Z_{\rm Riemann}(X,s): =   \sum_{|\gamma|<X}X^{i\gamma}/i\gamma^s,$$ where again $\gamma$ ranges through the nontrivial zeroes of the Riemann zeta function.  For fixed $X\ge 0$ the function $Z_{\rm Riemann}(X,s)$ extends to a meromorphic function of $s$ on the  complex plane, and for $X>0$ it is entire{\footnote{ See \cite{Fu3} for the latter statement, and \cite{Fu1}, \cite{Fu2} for its proof.\vskip10pt}}.  The special case of $Z_{\rm Riemann}(1,s)$  fits into the immense literature regarding `spectral zeta-functions,' that extends to asymptotic
distributions of eigenvalues for oscillating membranes, and  to Zeta-functions of Laplacians {\footnote{ For this, see \cite{W} and its bibliography, which {\it remains} a useful, and delightful,   thing to read!}}. As for the Gibbs phenomenon, Theorem 3 of \cite{Fu3} offers the following jump-discontinuity analysis (in the variable $X$) for the analytic continuation of the Dirichlet series $Z_{\rm Riemann}(X,s)$ at real points $0 < s = \sigma < 1$.

 $$\lim_{X \to p^k\pm 0}{{Z_{\rm Riemann}(X,\sigma)}{|\log X -\log(p^k)|^{1-\sigma}} \ = \ \mp {\frac{\log p}{2\pi p^{k/2}}}\int_0^{\infty}{\frac \sin t}{t^{\sigma}}}dt.$$
 \vskip10pt
% \begin{project} Rework this theory to cover the case of $S_E(X)$.\end{project}
  \vskip10pt
  Here is a small picture exhibition of the Gibbs phenomenon for our oscillatory terms $S_E(X)$.  These oscillatory terms appear roughly linear around---of course---the discrete jumps at powers of prime numbers, as we would expect from
  the explicit formula. [[remove some of these and/or put them in a grid]]
 \vskip10pt

 \centerline{ $E = 11a$  between $40$ and $60$}
   \vskip10pt
 \hskip100pt\includegraphics[width=0.6\textwidth]{plots/11a40t60.pdf}
    \newpage
  \centerline{ $E = 11a$  between $60$ and $100$}
   \vskip10pt
 \hskip100pt\includegraphics[width=0.6\textwidth]{plots/11a60t100.pdf}

   \vskip10pt
   \centerline{ $E = 11a$  between $990$ and $1010$}
   \vskip10pt
 \hskip100pt\includegraphics[width=0.6\textwidth]{plots/11a990t1010.pdf}

 \newpage
   \centerline{ $E = 37a$  between $40$ and $60$}

   \vskip10pt
 \hskip100pt\includegraphics[width=0.6\textwidth]{plots/37a40t60.pdf}

   \centerline{ $E = 37a$  between $60$ and $100$}
   \vskip10pt
 \hskip100pt\includegraphics[width=0.6\textwidth]{plots/37a60t100.pdf}
 \newpage
  \centerline{ $E = 37a$  between $990$ and $1010$}
   \vskip10pt
 \hskip100pt\includegraphics[width=0.6\textwidth]{plots/37a990t1010.pdf}
 \newpage
  \centerline{ $E = 389a$  between $40$ and $60$}
   \vskip10pt
 \hskip100pt\includegraphics[width=0.6\textwidth]{plots/389a40t60.pdf}

   \centerline{ $E = 389a$  between $60$ and $100$}
   \vskip10pt
 \hskip100pt\includegraphics[width=0.6\textwidth]{plots/389a60t100.pdf}
\newpage
  \centerline{ $E = 5077a$  between $40$ and $60$}
   \vskip10pt
 \hskip100pt\includegraphics[width=0.6\textwidth]{plots/5077a40t60.pdf}

   \centerline{ $E = 5077a$  between $60$ and $100$}
   \vskip10pt
 \hskip100pt\includegraphics[width=0.6\textwidth]{plots/507760t100.pdf}







%  Let $E$ be an elliptic curve over the field of rational numbers, and for all primes $p$ for which the reduction of $E$ modulo $p$ is an elliptic curve over the prime field ${\bf F}_p$  (this will happen for all but finitely many $p$) let $N_E(p):= |E({\bf F}_p)|$  be the number of points of the reduction of $E$ over ${\bf F}_p$.  To more easily compare $N_E(p)$ with the quantity $p+1$, put $a_E(p):= (p+1)-N_E(p)$ so that our estimate ($p+1$) is an ``over-count"  for the number of points of our elliptic curve $E$ mod $p$  if and only if  $a_E(p)$ is positive; and an ``under-count" if negative.

% In a letter  [\cite{S}]  to one of us (to B.M.) Peter Sarnak sketched reasons for the statements made about the three formats for sums of local data that we  discussed  above. As we  understand it, the computations in that letter was, at least in part, the fruit of conversations with Andrew Granville. We are grateful for that, and for  illuminating discussions with    Granville, Rubinstein, and Sarnak about this  phenomenon.  As already mentioned, assuming a list of standard conjectures about the behavior of $L$-functions, together with some very plausible but less standard conjectures, Sarnak begins by showing that $$X\mapsto {\frac{\log\ X}{\sqrt X}}\sum_{p \le X}{\frac{a_E(p)}{\sqrt p}}$$ has a limiting distribution with {\it mean} equal to $1- 2r(E)$ where $r(E)$ is the Mordell-Weil rank of the elliptic curve $E$.

 % The {\it variance} of this limiting distribution  is the sum of the squares of the reciprocals of the absolute values of the nonreal zeroes of the $L$-function of $E$. The argument for this follows Mike Rubenstein's and Peter Sarnak's line of reasoning in the article {\it Chebyshev's Bias} [\ref{R-S}]{\footnote{For another expository account of number theoretic issues related to biases, see [\ref{GM}].}}. If, however, we apply similar reasoning to the quantity specifically measuring the race depicted in our graphs; i.e.,  $$X\mapsto {\frac{\log\ X}{\sqrt X}}\big(\#\{ {p \le X};\ a_E(p) > 0\} \ - \ \#\{ {p \le X};\ a_E(p) < 0\}\big) $$ one computes  (given reasonable conjectures, and guesses) the {\it mean} and one discovers that it conforms fairly well with the data; the {\it variance}, however, is infinite, so whatever bias we see in our finite stretch of data will eventually wash out{\footnote{ This is specific to elliptic curves $E$ with no complex multiplication, as our examples below all are. The non-finiteness of the variance is related to the fact that the (expected) number of  zeroes---in  intervals  $(1/2, i/2+iT)$ ($T > 0$)---of the $L$ function of the $n$-th symmetric power of the newform $f_E$ attached to  $E$   grows at least linearly with $n$.}}.  In this section, and subsequent subsections, I will be simply transcribing---with minor notational modifications---a few extracts from a letter that Peter Sarnak wrote to me.

 \subsection{Preliminaries}

Let $E$ be an elliptic curve over ${\Q}$ without complex multiplication associated to a newform $f$ with Fourier expansion:
$$f(q) = q+\sum_{n\ge 2}a_E(n)q^n.$$


 %More specifically, the {\it means}  directly interpretable as {\it biases}  determine   the orders of vanishing at the central point of the $L$-functions of the symmetric powers of odd order of the modular eigenform attached to $E$ and vice versa..

 %  \vskip20pt \item {\it conditional biases?}  For example, given two elliptic curves $E_1, E_2$ over ${\Q}$ (that are not isogenous), say that a prime $p$ is of type $(+,+)$ if both $a_{E_1}(p)$ and $a_{E_2}(p)$ are positive, of type  $(+,-)$ if  $a_{E_1}(p)$ is  positive and $a_{E_2}(p)$ negative, etc.

 %Now  {\it race} the four types of primes against each other!  What is the ensuing statistics, and how much of the analytic number theory regarding zeroes of $L$ functions attached to  $$symm^m(f_{E_1})\otimes symm^n(f_{E_2})$$  do we need to compute biases, if  such biases exist? \end{enumerate}
\vskip10pt

\section{Some pictures}

[[Ideas:

- log scale

- also draw as a distribution of values.

]]

  \vskip30pt


     \subsection{\bf (Graphs of \ \   $X\mapsto {D}_E(X) = {\frac{1}{\log\ X}}\sum_{p \le X}{\frac{a_E(p)\log p}{ p}}$) }
      \vskip40pt

    \centerline{\bf Rank $r=0$:\ \ \  ${\mathcal E}=$11A.}

   \includegraphics[width=0.9\textwidth]{plots/even_smoother-rank0-4million.png}


 \vskip40pt


  \centerline{\bf Rank $r=1$:\ \ \  ${\mathcal E}=$37A.}


   \vskip20pt



    \includegraphics[width=0.9\textwidth]{plots/even_smoother-rank1-4million.png}{1.2}~\label{s37}
%

   \newpage


  \centerline{\bf Rank $r=2$:\ \ \  ${\mathcal E}=$389A.}


  \vskip20pt




    \includegraphics[width=0.9\textwidth]{plots/even_smoother-rank2-4million.png}{1.2}~\label{s389}


 \vskip20pt


  \centerline{\bf Rank $r=3$:\ \ \  ${\mathcal E}=$5077A.}


 \vskip20pt



    \includegraphics[width=0.9\textwidth]{plots/even_smoother-rank3-4million.png}{0.9}~\label{s5077}


  \vskip10pt

  \centerline{\bf Rank $r=4$.}


  \vskip20pt


    \includegraphics[width=0.9\textwidth]{plots/smooth-prime_race-rank4-4million.png}{0.9}~\label{sr4}

 \vskip40pt


  %\centerline{\bf Rank $r=5$.}


 %\vskip20pt


     %\includegraphics[width=0.9\textwidth]{plots/smooth-prime_race-rank5-4million.png}{1.2}~\label{sr5}
 \newpage

  \centerline{\bf Rank $r=6$.}


 \vskip20pt



    \includegraphics[width=0.9\textwidth]{plots/even_smoother-rank6-4million.png}{1.2}~\label{sr6}
         \subsection{\bf (Graphs of \ \   $X\mapsto {\mathcal D}_E(X) = {\frac{\log\ X}{\sqrt X}}\sum_{p \le X}{\frac{a_{\mathcal E}(p)}{\sqrt p}}$)}

 \vskip40pt


  \centerline{\bf Rank $r=0$:\ \ \  ${\mathcal E}=$11A.}
   \vskip20pt
  \includegraphics[width=0.9\textwidth]{plots/illustsmooth-11}{.8}~\label{s11}
  \

    \newpage

  \centerline{\bf Rank $r=1$:\ \ \  ${\mathcal E}=$37A.}


   \vskip40pt



    \includegraphics[width=0.9\textwidth]{plots/smooth-prime_race-37a-4million.png}{1.2}~\label{s37}
%

   \vskip40pt


  \centerline{\bf Rank $r=2$:\ \ \  ${\mathcal E}=$389A.}


  \vskip20pt


    \includegraphics[width=0.9\textwidth]{plots/smooth-prime_race-389a-4million.png}{0.9}~\label{s389}


\newpage

\
  \centerline{\bf Rank $r=3$:\ \ \  ${\mathcal E}=$5077A.}


 \vskip10pt



    \includegraphics[width=0.9\textwidth]{plots/smooth-prime_race-5077a-4million}{0.9}~\label{s5077}

  \vskip30pt


  \centerline{\bf Rank $r=4$.}


  \vskip10pt



    \includegraphics[width=0.9\textwidth]{plots/smooth-prime_race-rank4-4million.png}{0.9}~\label{sr4}

 \newpage


  \centerline{\bf Rank $r=5$.}


    \includegraphics[width=0.9\textwidth]{plots/smooth-prime_race-rank5-4million.png}{0.9}~\label{sr5}


  \vskip30pt


  \centerline{\bf Rank $r=6$.}


 \vskip20pt



    \includegraphics[width=0.9\textwidth]{plots/smooth-prime_race-rank6-4million.png}{1.2}~\label{sr6}
      \newpage
\subsection{\bf (Graphs of \ \   $X\mapsto \Delta_E(X)=  {\frac{\log\ X}{\sqrt X}}\#\{ p < X\ | \ a_E(p) > 0\}\ - \ \#\{ p < X\ | \  a_E(p) < 0\})$}  \vskip40pt


 \centerline{\bf Rank $r=0$:\ \ \  ${\mathcal E}=$11A.}~\includegraphics[width=0.9\textwidth]{plots/normalized_straight-prime_race-11a-4million.png}{.8}~\label{nr11}


\vskip40pt



  \centerline{\bf Rank $r=1$:\ \ \  ${\mathcal E}=$37A.}


 \vskip60pt



    \includegraphics[width=0.9\textwidth]{plots/normalized_straight-prime_race-37a-4million.png}{.8}~\label{nr37}
%

  \vskip40pt



  \centerline{\bf Rank $r=2$:\ \ \  ${\mathcal E}=$389A.}


  \vskip20pt



    \includegraphics[width=0.9\textwidth]{plots/normalized_straight-prime_race-389a-4million.png}{.8}~\label{nr389}


   \vskip60pt


  \centerline{\bf Rank $r=3$:\ \ \  ${\mathcal E}=$5077A.}




    \includegraphics[width=0.9\textwidth]{plots/normalized_straight-prime_race-5077a-4million.png}{.8}~\label{nr389}






\begin{thebibliography}{bib}
 \bibitem{B}\label{B} Bober, J.W.: Conditionally bounding ranks of elliptic curves, (ar?iv:1112.1503)
 \bibitem{C-S}\label{C-S} Conrey, J.B., Snaith, N.C.:  On the orthogonal symmetry of $L$-functions of powers of a Hecke character, arxiv.org/pdf/1212.2681


 \bibitem{DMW}\label{DMW} Dummigan N, Martin P, Watkins M.: Euler factors and local root numbers for symmetric powers of elliptic curves, Pure and Applied Mathematics, Quarterly, {\bf 5},  no. 4, 1311-1341  (2009)
  \bibitem{F}\label{F}  Fiorilli, D.: Elliptic curves of unbounded rank and Chebyshev's Bias, preprint (2012)
        \bibitem{Fu1}\label{Fu1} Fujii, A.:  Comment. Math. Univ. Sancti Pauli, {\bf 31}, 99-113 (1982)
        \bibitem{Fu2}\label{Fu2} Fujii, A.:  Comment. Math. Univ. Sancti Pauli, {\bf 32}, 229-248 (1983)
      \bibitem{Fu3}\label{Fu3} Fujii, A.: Zeroes, Eigenvalues and arithmetic, Proc. Japan Acad., {\bf 60} 22-25 (1984)
    \bibitem{Fu4}\label{Fu4} Fujii, A.: An additive problem of prime numbers, III, Proc. Japan Acad., {\bf 67} 278-283 (1991)
   \bibitem{G}\label{G} Guinand, A. P.: A summation formula in the theory of prime numbers, Proc. London Math. Soc., {bf 50}, 107-119 (1945)
 \bibitem{GM}\label{GM}  Granville, A., Martin, G.:  Prime number races,
American Mathematical Monthly {\bf 113} 1-33 (2006)
  \bibitem{MV}\label{MV} Montgomery, H.L, Vaughan, R.C.:   {\it Multiplicative Number Theory I: Classical Theory} (Cambridge Studies in Advanced Mathematics) Cambridge University (2007)
  \bibitem{R-S}\label{R-S}   Rubinstein, M., Sarnak, P.: Chebyshev's Bias,  Experimental  Mathematics {\bf 3}  (1994) 174-197.
    \bibitem{S}\label{S} Sarnak, P.: Letter to Barry Mazur on ``Chebyshev's bias" for $\tau(p)$, http://web.math.princeton.edu/sarnak/MazurLtrMay08.PDF  November (2007)
  \bibitem{M-N}\label{M-N} Martin, G., NG, N.: Nonzero values of Dirichlet $L$-functions in vertical arithmetic progressions,
    arXiv:1109.1788v2 [math.NT]  Aug (2012)

        \bibitem{M-W}\label{M-W} Martin, G., Watkins, M.:  Symmetric powers of elliptic curve $L$-functions, arxiv.org/pdf/math/0604095

     \bibitem{W}\label{W}  Weyl, H.:  Ramifications, old and new, of the eigenvalue problem,  Bull. Amer. Math. Soc. {\bf 56}, Number 2 115-139 (1950)
\end{thebibliography}

% Barry, I put this in so that we can use my file biblio.bib which has a ton of references.  We can copy bibtex citations from MathSciNet.  Then at the end we can export to the format you have above automatically...
\bibliographystyle{amsalpha}
\bibliography{biblio}


\end{document}

